    \subsection{Vigesima Septima}
      La disoluci\'on de la Asociaci\'on podr\'a ser acordada por una Asamblea
      Extraordinaria, la cual  decidir\'a el futuro de la misma por una
      mayor\i{}a de tres cuartas partes (3/4) de sus miembros activos. As\'i{}
      mismo el destino de los bienes de la Asociaci\'on, previa opini\'on de la
      Coordinaci\'on General, y por decisi\'on de la Asamblea convocada para tal
      fin, si se resolviere la liquidaci\'on patrimonial de la Asociaci\'on,
      este podr\'a ser donado a una o m\'as organizaciones con fines an\'alogos
      a los de la de la \emph{Asociaci\'on Civil gUsLA}.
      
    \subsection{Vigesima Octava}
      Los Estatutos de la Fundaci\'on podr\'an ser total o parcialmente
      modificados por una Asamblea Extraordinaria, convocada para tal fin, a
      solicitud  de por lo menos, el cincuenta por ciento (50\%) de sus miembros
      y con el voto favorable del setenta y cinco (75\%)  de los miembros
      activos presentes en la Asamblea. 
      
    \subsection{Vigesima Novena}
      El ejercicio econ\'omico de la Asociaci\'on comenzar\'a el primero de
      enero de cada a\~no y terminar\'a el treinta y uno de diciembre de ese
      mismo a\~no. 

    \subsection{Trigesima}
      La Asociaci\'on no se identificar\'a con partido o grupo pol\'itico
      alguno, raza, credo, religi\'on, condici\'on social, ni trabajar\'a en
      funci\'on determinada de uno o alguno de ellos; trabajar\'a solamente por
      lo establecido en sus objetivos. 
